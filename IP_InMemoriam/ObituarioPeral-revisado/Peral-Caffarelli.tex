%% !TEX encoding = UTF-8 Unicode
%
%% !TEX root = ObiStein.tex
%
%%---------------------
%\title{Ireneo Peral en la memoria}
%\author{Luis Caffarelli} 
%%---------------------
%
%%---------------------
%\contact{Luis Caffarelli, Department of Mathematics, University of Texas, Austin}
%{caffarel@math.utexas.edu}{}
%%---------------------
%
%%---------------------
%\makesemititle
%%---------------------

\subsection{Ireneo Peral en la memoria}

\begin{center}\large
\textbf{Luis Caffarelli}\\[1em] 
University of Texas at Austin, EE.UU.\\
{\color{azulsema}\rule{.5\linewidth}{1pt}}
\end{center}

\vskip 5mm


Nac\'i y crec\'i en Buenos Aires una ciudad muy conectada con Espa\~na.
Despu\'es de la secundaria fui admitido en el Centro de Ciencias de la Universidad de Buenos Aires,  que albergaba, entre otras, secciones de Matem\'aticas, F\'isica y Qu\'imica. Entre los profesores que impart\'ian cursos avanzados hab\'ia varios de un alto nivel cient\'ifico que hab\'ian emigrado de Espa\~na con motivo de la Guerra Civil.
Gracias a estos tuve la oportunidad de recibir una educaci\'on cient\'ifica muy s\'olida. 
Entre ellos se encontraba Luis Santal\'o, al que recuerdo con mucho cari\~no, que me ayud\'o a avanzar en geometr\'ia y an\'alisis. 



Realic\'e el doctorado en Buenos Aires y obtuve una beca posdoctoral para ir a Estados Unidos.
En ese momento en Europa y EE. UU. exist\'ia una red internacional de cient\'ificos y, en particular, j\'ovenes y excelentes matem\'aticos en Espa\~na involucrados con el An\'alisis Real y sus aplicaciones.
A trav\'es de la Universidad Men\'endez Pelayo en Espa\~na y varias instituciones en los Estados Unidos integradas con ella (Instituto de Estudios Avanzados, el Instituto Courant, la Universidad de Chicago) desarroll\'e investigaciones comunes y conexiones con varios de estos matem\'aticos,  como Juan Luis V\'azquez, Antonio C\'ordoba, Fernando Soria o Rafael de la Llave.
Y entre ellos tambi\'en  con Ireneo, un matem\'atico igualmente brillante.



Cuando lo conoc\'i, \'el era joven y, sin embargo, ya me pareci\'o un cient\'ifico con un profundo conocimiento de las propiedades estructurales del An\'alisis. 
Su trabajo a lo largo del tiempo ha consistido en estudiar ecuaciones que describen fen\'omenos naturales por un lado y que requieren estructuras matem\'aticas muy complejas por otro, lo que implica un esfuerzo considerable para superar esa complejidad. Personalmente siempre tuve  una gran admiraci\'on por la comprensi\'on matem\'atica que pose\'ia en su trabajo.
Ireneo ten\'ia una forma natural de avanzar a trav\'es de las dificultades de un problema matem\'atico confrontando entre s\'i cada una de las posibles l\'ineas estructurales que se pod\'ian presentar. De hecho, su trabajo es muy valorado como lo atestigua, por ejemplo, el gran n\'umero de menciones que ha recibido en sus art\'iculos y que se pueden consultar en MathSciNet. 
No solo su obra tuvo cientos de citas, sino que estas citas han sobrevivido al paso del tiempo y aparecen en art\'iculos actuales de revistas fundamentales.
Para m\'{\i} fue una enorme satisfacci\'on que Ireneo, con el que ya hab\'ia empezado a trabar una gran amistad, visitara el Instituto de Estudios Avanzados. De esta forma tuvimos  la oportunidad de desarrollar  ideas y nuevos m\'etodos, algunos de los mejores  que  he encontrado en mi trabajo.
Puedo decir que estoy muy orgulloso de esa colaboraci\'on que pude mantener con \'el.



Para completar el retrato de Ireneo en este breve escrito, me queda agregar su profundo sentido de humanidad.
Como ya he dicho, hab\'iamos desarrollado a lo largo de estos a\~nos una gran amistad compartiendo momentos agradables, expectativas comunes, celebraciones de felicidad.
De hecho, nuestro \'ultimo recuerdo es de hace unos meses, de visita en Madrid, disfrutando de una cena llena de felicidad con \'el y su esposa, Magdalena Walias, \'Angeles Encinar y su esposo Fernando Soria e Irene Gamba, mi esposa.
Todo discurr\'ia en un ambiente pl\'acido, y en nuestra conversaci\'on nuestras matem\'aticas eran el presente, nuestros hijos eran el futuro.
\'El quedar\'a grabado en nuestra memoria para siempre.

%
%\printcontact
%
%\endinput
%
%%---------------------------

\begin{center}
%\mbox{}\\[.1ex]
\resizebox{.5\linewidth}{!}{\color{azulsema}\rule{.5\linewidth}{1pt}
{\large $\diamond$} {\huge $\diamond$} {\large $\diamond$} \rule{.5\linewidth}{1pt}}
\end{center}
