%% !TEX encoding = UTF-8 Unicode
%
%% !TEX root = ObiStein.tex
%
%%---------------------
%\title{\lq\lq Per Ireneo\rq\rq}
%\author{Lucio Boccardo} 
%%---------------------
%
%%---------------------
%\contact{Lucio Boccardo, Sapienza Universit\`a di Roma}
%{boccardo@mat.uniroma1.it}{}
%%---------------------
%
%%---------------------
%\makesemititle
%%---------------------

\subsection{\guillemotleft Per Ireneo\guillemotright }

%\footnote{Para Ireneo, en Italiano {(quiz\'as no haga falta poner esta aclaraci\'on)}}

\begin{center}\large
\textbf{Lucio Boccardo}\\[1em] 
Sapienza Universit\`a di Roma, Italia\\
{\color{azulsema}\rule{.5\linewidth}{1pt}}
\end{center}

\vskip 5mm


Ireneo, esta ma\~nana vendr\'e  de nuevo a la  Aut\'onoma.   

En la quinta planta, sobre la puerta de tu despacho, leer\'e <<Sed breves con las malas noticias>>.

Hoy no tengo malas noticias. La \'ultima vez que vine, ten\'{\i}a una:
Ireneo, no hemos ganado la Medalla Fields; t\'u me respondiste:
la pr\'oxima vez intentaremos ser m\'as j\'ovenes.

Hoy no tengo malas noticias. Empezaremos trabajando en la pizarra; despu\'es de un rato, me llevar\'as a la cafeter\'{\i}a para un \textit{cappuccino}: t\'u esperas que el caf\'e del \textit{cappuccino} haga de m\'{\i} un coautor de nivel Medalla Fields; no es verdad, pero me alegran el \textit{cappuccino} y las charlas amistosas.
 
Hoy no tengo malas noticias, pero te desvelar\'e un secreto: fui el \textit{referee} de tu art\'iculo sobre Hardy; y estoy contento de haberlo sido. Pues continuaremos trabajando, como hacemos cuando estamos juntos: con serenidad, pasi\'on por las EDP   y  con amistad entre nosotros.

Ireneo, estoy subiendo a la quinta planta.

%\printcontact
%
%\endinput
%
%%---------------------------

\begin{center}
\mbox{}\\[.1ex]
\resizebox{.5\linewidth}{!}{\color{azulsema}\rule{.5\linewidth}{1pt}
{\large $\diamond$} {\huge $\diamond$} {\large $\diamond$} \rule{.5\linewidth}{1pt}}
\end{center}
